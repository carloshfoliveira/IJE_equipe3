\documentclass[11pt]{article}

\usepackage[utf8]{inputenc}
\usepackage[brazil]{babel}
 
\begin{document}
\title{TerraCota}
\author{NullPointer Corporation}
\date{}
\maketitle

\newpage

\tableofcontents
\newpage
\part{Apresentação e Resumo do jogo}
\section{Apresentação}
TerraCota é um jogo do tipo aventura com uso de puzzle para progresso na história do jogo. 

\section{Resumo do jogo}
Por conta de um desastre iminente, a humanidade construiu diversas torres auto-suficientes para se abrigar.
Centenas de anos depois, o mundo já re-equilibrado, a sociedade "evoluiu" a falar por meio de símbolos, ao invés de sons e palavras, e retornou às culturas de seus ancestrais.

Em uma das torres vive Inti, numa sociedade que esqueceu da existência do mundo de fora de sua torre.
Um dia, por algum motivo, ele acaba encontrando a saída da torre, e ao descobrir que existe algo além da torre, decide explorar.

Ele encontra Killa sentada nos galhos de uma árvore, e ela misteriosamente possui um interesse grande por ele.
\newpage
\part{Principais Características}
\newpage
\part{Público e plataformas alvo}
\section{Público}
O jogo é voltado para o público geral por não possuir nenhum conteúdo que o faz ser censurado a nenhuma idade.
\section{Plataformas alvo}
As plataformas suportadas, inicialmente, serão Windows e Linux.
O Linux possui um grande suporte de bibliotecas e ferramentas de desenvolvimento de software e o Windows possui uma grande base de usuários jogadores, fazendo com que ambas os sistemas operacionais se tornem alvos para o jogo.
\newpage
\part{Relação dos logotipos}
\section{Conceito de logotipo}

\newpage
\part{Equipe}
\section{Contato por e-mail}
A equipe pode ser contatada através do email: \textit{nullpointercorporation@gmail.com}

\section{Contato por repositório}
Caso a intenção seja contatar a equipe para reportar algum \textit{bug} no jogo, o endereço a seguir deve ser acessado: https://github.com/nullpointercorporation/terracota/issues/
\newpage
\end{document}

