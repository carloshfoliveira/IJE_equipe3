\documentclass[12pt]{article}

\usepackage[utf8]{inputenc}
\usepackage[brazil]{babel}
\usepackage{graphicx}
\usepackage[table]{xcolor}
\usepackage{tabularx}
\usepackage{hyperref}
\usepackage{booktabs}
\usepackage{bookmark}

\begin{document}
\title{TerraCota\\Game Design Document}
\author{NullPointer Corporation}
\date{v0.1}
\maketitle

\newpage

\begin{table}[h]
  \centering
  \begin{tabular}{ccll}
    \toprule
    \textbf{Versão} & \textbf{Data} & \textbf{Autor} & \textbf{Descrição} \\
    \midrule
    0.1 & 22/04/2015 & Carlos Oliveira  & Versão Inicial \\
    \rowcolor[gray]{0.9}
    \bottomrule
  \end{tabular}
  \caption{Histórico de Revisões}
\end{table}

\newpage

\tableofcontents

\newpage

\section{Objetivo do jogo}

\subsection{Conceito do jogo}
O jogo terá uma movimentação semelhante a games do estilo beat'em up (briga de
rua em português), como principais referências e exemplos os jogos Little
Fighter 2 (PC), Captain Commando (SNES), Double Dragon (SNES).

A temática será a de um mundo pós-apocalíptico, onde várias tecnologias se
extinguiram e a humanidade volta a comunicar-se de forma arcaica por meio de
figuras e indicações corporais. O jogador terá que, entre outros desafios,
resolver puzzles decifrando códigos audiovisuais para completar as quests.

Todos os membros da equipe estão cientes do conceito do jogo. No diretório
deste arquivo é possível visualizar algumas imagens feitas pelo artista Pedro
Braga que podem auxiliar no entendimento da temática.

\subsection{Principais características}
Uma das características mais interessantes do jogo são o modo como os gêneros
aventura e puzzle interagem entre si, fazendo com o que o jogador precise
pensar e decifrar qual será sua missão, pois as informações serão dadas
através de símbolos.

A movimentação é em dois eixos, vertical e horizontal, mais do que um jogo de
plataforma, que contém só um, desconsiderando a função de pular, também é uma
característica diferenciada do jogo, pois traz um modo de exploração em visão
em múltiplos planos. A inspiração desse estilo de movimentação veio,
principalmente, do jogos Capitão Comando, Little Fighter 2 e Double Dragon.

\section{Visão geral}
O jogo é dividido em 5 segmentos, ou 5 capítulos:
\begin{enumerate}
\item Tutorial pra interação com o mundo, tarefas simples como "ir falar com o carteiro" ou "buscar o leite na despensa" para o jogador se acostumar com a linguagem do jogo.
Após essa pequena tarefa é revelado que alguns morcegos/monstros invadiram o último andar da torre, e o jogador deve ir se livrar deles.
\item O jogador precisa de alguma arma, então deve ir até o ferreiro para conseguir uma espada simples. Ele sobe até o último andar da torre e elimina os morcegos, mas por acidente desperta o mecanismo de defesa da torre, que o ataca (chefão 1).
Durante a batalha, a espada do Inti se prova ineficaz (pode talvez quebrar) porém ele acha uma espada fincada na parede próxima ao chefão. Ele remove a espada (o que destranca a porta para a saída pro mundo afora) e usa ela pra derrotar o chefão.
Ao derrotar o chefão uma porta atrás dele se revela e abre, revelando um mundo inteiro do lado de fora.
\item Curioso pra explorar o mundo, Inti sobe no "elevador" do lado de fora da torre e desce para o chão, entrando numa floresta.
Ao caminhar pela floresta ele encontra a Killa, sentada em um galho. Ela percebe ele, sorri, chama ele e sai andando. A missão se torna seguir a Killa.
Ela guia ele até a aldeia dela, e introduz o pai dela (cacique) e o jogador se torna livre a explorar os arredores e a realizar sidequests (se der tempo de desenvolvê-las), a missão se torna retornar à torre e contar para os pais sobre a existência de mais pessoas além das que vivem na torre com ele.
\item Ele retorna para a torre, Killa fica para trás porque o pai dela não deixa ela sair da floresta, e tenta convencer as pessoas sobre o mundo de fora. Ninguém acredita nele nem segue ele quando ele tenta mostrar. Ele decide procurar um artefato valioso pra provar a existência do mundo lá fora (estamos pensando em fazer um artefato que seja presente em toda torre, porém só um. Ao trazer um de outra torre ele prova a existência dela)
Então ele volta para a floresta e Killa se junta a ele mais uma vez, e ajuda ele a chegar na torre onde ela vivia. Numa região alagada da floresta. Os dois entram na torre.
\item A torre abandonada, cheia de puzzles. Ao chegar na câmara onde o artefato está, eles percebem que precisam de uma chave pra destrancar ela. Quando eles encontram a chave, o chefão a engole antes que eles consigam pegar ela. O objetivo se torna derrotar o chefão pra pegar a chave de volta, e então conseguir o artefato.
Com o artefato em mãos, o Inti volta pra torre, prova a existência do mundo afora, e o jogo termina com a população na frente da porta, olhando o mundo do lado de fora
\end{enumerate}
\section{Controles}

\subsection{Lista de movimentos}
O personagem poderá se movimentar da seguinte maneira:

\begin{itemize}
    \item Baixo
    \item Cima
    \item Direita
    \item Esquerda
    \item Interação
    \begin{itemize}
        \item Ação especial
        \begin{itemize}
            \item Empurrar bloco; ou
            \item Escalar
        \end{itemize}
        \item Conversar com NPC
        \item Pegar item
    \end{itemize}
    \item Ataque
\end{itemize}

\subsection{Esquema básico}

[W] [A] [S] [D] - movimentação

[ESC] - Sair do jogo(no menu)

[E] [N] - interagir

[SPACE] - Atacar

[C] - Trocar personagem

[ESC] [P] - Pause(durante jogo)

\end{document}

