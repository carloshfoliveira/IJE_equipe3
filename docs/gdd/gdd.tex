\documentclass[12pt]{article}

\usepackage[utf8]{inputenc}
\usepackage[brazil]{babel}
\usepackage{graphicx}
\usepackage[table]{xcolor}
\usepackage{tabularx}           
\usepackage{hyperref}
\usepackage{booktabs}
\usepackage{bookmark}

\begin{document}
\title{TerraCota\\Game Design Document}
\author{NullPointer Corporation}
\date{v0.1}
\maketitle

\newpage

\tableofcontents

\newpage

\begin{table}[h]
  \centering
  \begin{tabular}{ccll}
    \toprule
    \textbf{Vers�o} & \textbf{Data} & \textbf{Autor} & \textbf{Descri��o} \\
    \midrule
    0.1 & 22/04/2015 & Carlos Oliveira  & Vers�o Inicial \\
    \rowcolor[gray]{0.9}
    \bottomrule
  \end{tabular}
  \caption{Hist�rico de Revis�es}
\end{table}


\newpage

\section{Objetivo do jogo}

\subsection{Conceito do jogo}

\subsection{Principais caracter��sticas}
Uma das caracter�sticas mais interessantes do jogo s�o o modo como os g�neros aventura e puzzle interagem entre si, fazendo com o que o jogador precise pensar e decifrar qual ser� sua miss�o, pois as informa��es ser�o dadas atrav�s de s�mbolos.

A movimenta��o � em dois eixos, vertical e horizontal, mais do que um jogo de plataforma, que cont�m s� um, desconsiderando a fun��o de pular, tamb�m � uma caracter�stica diferenciada do jogo, pois traz um modo de explora��o em vis�o em m�ltiplos planos. A inspira��o desse estilo de movimenta��o veio, principalmente, do jogos Capit�o Comando, Little Fighter 2 e Double Dragon.


\section{Vis�o geral}

\section{Controles}

\subsection{Lista de movimentos}

\subsection{Esquema b�sico}

\end{document}
