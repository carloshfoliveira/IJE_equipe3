\documentclass[12pt]{article}

\usepackage[utf8]{inputenc}
\usepackage[brazil]{babel}
\usepackage{graphicx}
\usepackage[table]{xcolor}
\usepackage{tabularx}
\usepackage{hyperref}
\usepackage{booktabs}
\usepackage{bookmark}

\begin{document}
\title{TerraCota\\Game Design Document}
\author{NullPointer Corporation}
\date{v0.1}
\maketitle

\newpage

\begin{table}[h]
  \centering
  \begin{tabular}{ccll}
    \toprule
    \textbf{Versão} & \textbf{Data} & \textbf{Autor} & \textbf{Descrição} \\
    \midrule
    0.1 & 22/04/2015 & Carlos Oliveira  & Versão Inicial \\
    \rowcolor[gray]{0.9}
    \bottomrule
  \end{tabular}
  \caption{Histórico de Revisões}
\end{table}

\newpage

\tableofcontents

\newpage

\section{Objetivo do jogo}

\subsection{Conceito do jogo}
O jogo terá uma movimentação semelhante a games do estilo beat'em up (briga de
rua em português), como principais referências e exemplos os jogos Little
Fighter 2 (PC), Captain Commando (SNES), Double Dragon (SNES).

A temática será a de um mundo pós-apocalíptico, onde várias tecnologias se
extinguiram e a humanidade volta a comunicar-se de forma arcaica por meio de
figuras e indicações corporais. O jogador terá que, entre outros desafios,
resolver puzzles decifrando códigos audiovisuais para completar as quests.

Todos os membros da equipe estão cientes do conceito do jogo. No diretório
deste arquivo é possível visualizar algumas imagens feitas pelo artista Pedro
Braga que podem auxiliar no entendimento da temática.

\subsection{Principais características}
Uma das características mais interessantes do jogo são o modo como os gêneros
aventura e puzzle interagem entre si, fazendo com o que o jogador precise
pensar e decifrar qual será sua missão, pois as informações serão dadas
através de símbolos.

A movimentação é em dois eixos, vertical e horizontal, mais do que um jogo de
plataforma, que contém só um, desconsiderando a função de pular, também é uma
característica diferenciada do jogo, pois traz um modo de exploração em visão
em múltiplos planos. A inspiração desse estilo de movimentação veio,
principalmente, do jogos Capitão Comando, Little Fighter 2 e Double Dragon.

\section{Visão geral}

\section{Controles}

\subsection{Lista de movimentos}
O personagem poderá se movimentar da seguinte maneira:

\begin{itemize}
    \item Frente
    \item Trás
    \item Direita
    \item Esquerda
    \item Pulo
    \item Ataque
\end{itemize}

\subsection{Esquema básico}

\end{document}
